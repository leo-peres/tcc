\chapter{Análise de Fourier de Funções Booleanas} \label{funcoes_booleanas}

Uma função Booleana é uma função $f: \binalphn \to \binalph$, porém nesta seção nós vamos na maior parte das vezes considerar funções com domínio $\{-1, 1\}^{n}$ e contradomínio $\{-1, 1\}$. O motivo para isso é que os resultados desta seção são geralmente mais intuitivos quando consideramos $\{-1, 1\}$ ao invés de $\{0, 1\}$. Para passar uma string $x \in \binalphn$ para uma string $x^{\prime} \in \pmonen$ usamos a transformação $x^{\prime}_{i} = (-1)^{x_{i}}$, então para cada função $f: \pmonen \to \pmone$ temos a função $f^{*}: \binalphn \to \binalph$ onde $f^{*}(x) = 1 - 2f(x^{\prime})$ que é ''equivalente`` à $f$ e também estruturalmente semelhante como vamos ver.

Alguém interessado em saber mais sobre a Análise de funções Booleanas pode ler o livro do Ryan O'Donnell~\cite{o2014analysis}.

Qualquer função $f: \pmonen \to \mathbb{R}$ pode ser representada da seguinte forma:

\begin{equation} \label{eq: interpol}
f(x) = \sum_{x^{\prime} \in \pmone} g(x^{\prime}, x)f(x^{\prime})
\end{equation}

Onde $g(x^{\prime}, x) = 1$ quando $x^{\prime} = x$ e 0 quando $x^{\prime} \neq x$, e é fácil verificar que $g(x^{\prime}, x) = \prod_{i = 1}^{n}\frac{1}{2}(1 + x_{i}x_{i}^{\prime})$ satisfaz exatamente isso. Seja $\mathcal{V}$ o espaço vetorial de todas funções de $\pmonen$ para $\mathbb{R}$ com produto interno $\langle f, g \rangle = 2^{-n}\sum_{x \in \pmonen}f(x)g(x) = \uniE[f(x)g(x)]$, nós queremos mostrar que as \emph{funções paridade} $\chi_{S}(x) = \prod_{i \in S}x_{i}$, $S \subseteq [n]$ e com $\chi_{\emptyset} = 1$, formam uma base de $\mathcal{V}$. Ou seja, podemos escrever $f$ como

\begin{equation} \label{eq: fexpan}
f(x) = \sum_{S \subseteq[n]}\widehat{f}(S)\chi_{S}(x)
\end{equation}

Onde $\widehat{f}(S)$ é a coordenada de $f$ na ''direção $S$``, o que nós vamos chamar de coeficiente de Fourier em $S$ de $f$.

\begin{prop}
Seja $f: \pmonen \to \mathbb{R}$. Podemos escrever $f$ como \ref{eq: fexpan} onde para cada $S \subseteq [n]$, $\widehat{f}(S) = \langle f, \chi_{S} \rangle$.
\end{prop}

\begin{proof}

Nós expandimos \ref{eq: interpol} com $g(x, x^{\prime}) = \prod_{i = 1}^{n}\frac{1}{2}(1 + x_{i}x_{i}^{\prime})$.

\begin{IEEEeqnarray*}{rCl}
    f(x) & = & \sum_{x^{\prime} \in \pmonen}f(x^{\prime})\prod_{i = 1}^{n}\frac{1 + x_{i}x_{i}^{\prime}}{2} \\
         & = & \sum_{x^{\prime} \in \pmonen} \sum_{S \subseteq [n]} 2^{-n} f(x^{\prime})\chi_{S}(x^{\prime})\chi_{S}(x) \\
         & = & \sum_{S \subseteq [n]} \chi_{S}(x) \Bigg(2^{-n}\sum_{x^{\prime} \in \pmonen} f(x^{\prime})\chi_{S}(x^{\prime})\Bigg) \\
         & = & \sum_{S \subseteq [n]} \langle f, \chi_{S} \rangle \chi_{S}(x)
\end{IEEEeqnarray*}

Então, para cada $S \subseteq [n]$, fazemos $\widehat{f}(S) = \langle f, \chi_{S} \rangle$ e obtemos \ref{eq: interpol}.

\end{proof}

Além disso, como existem $2^{n}$ funções paridades temos que elas formam uma base de $\mathcal{V}$. A base formada pelas funções paridade é ortornomal, basta observar que $\uniE[\chi_{S}(x)] = 0$ para qualquer $S \subseteq [n]$ que não seja $\emptyset$ e $\chi_{S}\chi_{S^{\prime}}$ é uma função paridade diferente de $\chi_{\emptyset}$ (mais especificamente, $\chi_{S \triangle S^{\prime}}$) sempre que $S$ e $S^{\prime}$ não são iguais. Quando $S = S^{\prime}$ temos $\uniE[\chi_{S}(x)\chi_{S}(x)] = \uniE[\chi_{\emptyset}] = 1$.

Nós denotamos por $\lVert f \rVert_{2}$ o valor $\sqrt{\langle f, f \rangle}$, e em geral $\lVert f \rVert_{p} = \uniE[\vert f(x) \vert^{p}]^{1/p}$.

Note que

\begin{IEEEeqnarray*}{rCl}
    \UniE[f(x)g(x)] & = & \UniE[\Bigg(\sum_{S \subseteq [n]}\widehat{f}(S)\chi_{S}(x) \Bigg)g(x)] \\
                & = & \sum_{S \subseteq [n]} \widehat{f}(S) \UniE[\chi_{S}(x)g(x)] \\
                & = & \sum_{S \subseteq [n]} \widehat{f}(S)\widehat{g}(S)
\end{IEEEeqnarray*}

Este resultado é o \emph{teorema de Plancherel}. Do teorema de Plancherel podemos obter o \emph{teorema de Parseval}: $\uniE[f(x)^{2}] = \sum_{S \subseteq [n]} \widehat{f}(S)^{2}$. No caso especial em que $f: \{-1, 1\}^{n} \to \{-1 ,1\}$ segue do teorema de Parseval que $\sum_{S \subseteq [n]} \widehat{f}(S)^{2} = 1$.

O valor esperado $\uniE[f(x)]$ de $f: \pmonen \to \mathbb{R}$ é igual a $\widehat{f}(\emptyset)$. Isto segue pela fórmula do coeficiente de Fourier: $\uniE[f(x)] = \uniE[f(x).1] = \uniE[f(x)\chi_{\emptyset}] = \widehat{f}(\emptyset)$. A variância de $f$ é $Var[f] = \E[f^{2}] - \E[f]^{2} = \sum_{S \neq \emptyset} \widehat{f}(S)^{2}$. O peso de $f$ em $k$, onde $0 \leq k \leq n$, é $\sum_{S \subseteq [n], \vert S \vert = k} \widehat{f}(S)^{2}$ que denotamos por $W^{k}[f]$. Daí podemos reescrever a fórmula para a variância de $f$ como $Var[f] = \sum_{k > 0}W^{k}[f]$. Nós iremos dizer que $f: \pmonen \to \pmone$ é balanceada se $\E_{x \sim \pmonen}[f(x)] = \widehat{f}(\emptyset) = 0$, se dissermos que $f$ é essencialmente balanceada nós geralmente queremos dizer que $f$ não está ``próxima'' de uma das funções constantes, o que pode significar algo como $1/10 \leq \Pr_{x \sim \pmonen}[f(x) = -1] \leq 9/10$ (ou, $\lvert \widehat{f}(\emptyset) \rvert \leq 4/5)$. Se $f: \binalphn \to \binalph$ então $f$ é balanceada sse $\E_{x \sim \binalphn}[f(x)] = \widehat{f}(\emptyset) = 1/2$, e essencialmente balanceada é definida analogamente ao caso $\pmone$.  O grau de $f$, que denotamos por $deg(f)$, é o maior valor $d \in [n]$ tal que $W^{d}[f] > 0$.

\subsubsection{Influência individual e total}

Nós dizemos que uma variável $i$ é \emph{pivotal} para uma string $x \in \pmone$ em $f$ se $f(x) \neq f(x^{\oplus i})$, onde $x^{\oplus i}$ é a string $x$ com a $i$-ésima coordenada invertida. 

\begin{defi}(Influência individual) \label{defi: inf}

A influência de $i$ em $f$, denotada por $Inf_{i}[f]$, é:

\begin{equation*}
    \uniPr[i \text{ é pivotal para } x \text{ em f }].
\end{equation*}

Ou seja,

\begin{equation*}
    Inf_{i}[f] = \uniPr[f(x) \neq f(x^{\oplus i})]
\end{equation*}

\end{defi}

Para funções Booleanas $f: \{-1, 1\}^{n} \to \{-1, 1\}$ nós fazemos a seguinte definição.

\begin{defi}

Seja $f: \{-1, 1\}^{n} \to \{-1, 1\}$. A derivada na direção $i$ de $f$ é definida como

\begin{equation*}
    D_{i}f(x) = \frac{f(x^{(i \rightarrow 1)}) - f(x^{(i \rightarrow - 1)}) }{2}
\end{equation*}

onde $x^{(i \rightarrow b)}$ é a string $x$ com a coordenada $i$ ''forçada`` como $b$.

\end{defi}

Note que para todo $i \in [n]$, $D_{i}f(x)^{2}$ é 1 quando a coordenada $i$ é pivotal para $x$ em $f$ e 0 caso contrário, portanto $\E[D_{i}f(x)^{2}] = Inf_{i}[f]$ e podemos generalizar a noção de influência para funções $g: \pmonen \to \mathbb{R}$ definindo $Inf_{i}[g] = \E[D_{i}g(x)^{2}]$.

\begin{prop} \label{prop: derivative_inf}

Seja $f: \pmonen \to \mathbb{R}$ e $i \in [n]$, então:

\begin{enumerate}

    \item $D_{i}f(x) = \sum_{S \ni i} \widehat{f}(S) \chi_{S \setminus \{i\}}(x)$;

    \item $Inf_{i}[f] = \sum_{S \ni i} \widehat{f}(S)^{2}$.

\end{enumerate}

\end{prop}

\begin{proof}

Primeiros nós provamos (1). Como $D_{i}$ é um operador linear, precisamos apenas mostrar que $D_{i}\chi_{S}$ é igual a $\chi_{S \setminus \{i\}}$ quando $i \in S$ e 0 caso contrário.

Se $i \in S$ então $\chi_{S}(x^{(i \rightarrow 1)}) = 1\times \prod_{i \in S \setminus \{i\}}x_{i} = \chi_{S \setminus \{i\}}(x)$. Na mesma forma vemos que $\chi_{S}(x^{(i \rightarrow -1)}) = -\chi_{S \setminus \{i\}}(x)$. Portanto

\begin{IEEEeqnarray*}{rCl}
    D_{i}\chi_{S}(x) & = & \frac{1}{2}(\chi_{S \setminus \{i\}}(x) - (-\chi_{S \setminus \{i\}}(x))) \\
                     & = & \frac{1}{2}(2\chi_{S \setminus \{i\}}(x)) \\
                     & = & \chi_{S \setminus \{i\}}(x)
\end{IEEEeqnarray*}

Por outro lado, se $i \notin S$ então $\chi_{S}(x^{(i \rightarrow 1)}) = \chi_{S}(x^{(i \rightarrow -1)})$ e portanto $D_{i}\chi_{S}(x) = 0$. O item (2) é só uma aplicação do teorema de Parseval e $Inf_{i}[f] = E[D_{i}f(x)^{2}]$.

\end{proof}

Nós podemos meio que generalizar o operador $D_{i}$ para funções $f: \binalphn \to \mathbb{R}$ com o operador de Laplace definido a seguir.

\begin{defi} (Operador de Laplace)

Seja $f: \binalphn \to \mathbb{R}$. O operador de expectativa $E_{i}$ é definido como

\begin{equation*}
E_{i}f(x) = \E_{b \in \binalph}[f(x^{(i \rightarrow b)})]
\end{equation*}

O operador de Laplace $L_{i}$ é

\begin{equation*}
L_{i}f = f - E_{i}f
\end{equation*}

\end{defi}

Ou seja, o operador de Laplace subtrai de $f$ a parte de $f$ que não depende da $i$-ésima coordenada, então parece razoável medir a ''importância`` da $i$-ésima coordenada usando $L_{i}$, assim como fizemos com $D_{i}$. Nós podemos provar o seguinte.

\begin{prop}

Seja $f: \binalph \to \mathbb{R}$, então

\begin{equation*}
L_{i}f = \sum_{S \ni i} \widehat{f}(S)\chi_{S}
\end{equation*}

\end{prop}

\begin{proof}

Para isso precisamos apenas mostrar que

\begin{equation} \label{eq: Ei_fexpan}
E_{i}f = \sum_{S \not\ni i}\widehat{f}(S)\chi_{S}
\end{equation}

e portanto o resultado segue pela definição do operador de Laplace.

Para mostrar que a fórmula \ref{eq: Ei_fexpan} é verdadeira nós só precisamos considerar o caso em que $f$ é uma das funções paridades por $E_{i}$ se tratar de um operador linear. Então, seja $S \subseteq [n]$. Se $S = \emptyset$ então $E_{i}\chi_{\emptyset} = 1 = \chi_{\emptyset}$. Se $S \neq \emptyset$:

\begin{IEEEeqnarray*}{rCl}
    E_{i}\chi_{S}(x) & = & \E_{b \in \binalph}[\chi_{S}(x^{(i \rightarrow b)})] \\
                     & = & \frac{1}{2}\big( \chi_{S}(x^{(i \rightarrow 0)}) + \chi_{S}(x^{(i \rightarrow 1)}) \big)
\end{IEEEeqnarray*}

Se $i \in S$ então $\chi_{S}(x^{(i \rightarrow 0)}) + \chi_{S}(x^{(i \rightarrow 1)})$ = 0 e caso contrário é igual a $2\chi_{S}(x)$, e portanto

\begin{equation*}
E_{i}\chi_{S} = \begin{cases}
                    \chi_{S} \text{ se } i \notin S \\
                    0 \text{ caso contrário}
                \end{cases}
\end{equation*}

\end{proof}

Note que se $f: \{-1, 1\}^{n} \to \mathbb{R}$ então $L_{i}f = x_{i}D_{i}f$ e portanto $\E[L_{i}f^{2}] = \E[D_{i}f^{2}] = Inf_{i}[f]$. Para funções $f: \binalphn \to \binalph$ temos somente que $E[L_{i}f^{2}] = \frac{1}{4}Inf_{i}[f]$, se adaptarmos a forma como influências individuais foram definidas em \ref{defi: inf}.

\begin{prop}
Seja $f: \binalphn \to \binalph$ então $\E[L_{i}f^{2}] = \frac{1}{4}\Pr_{x \sim \binalphn}[f(x) \neq f(x^{\oplus i})]$.
\end{prop}

\begin{proof}

Primeiro note que $\Pr_{x \in \binalphn}[f(x) \neq f(x^{\oplus i})] = \E_{x \in \binalphn}\big[\big(f(x) - f(x^{\oplus i})\big)^{2}\big]$. Portanto

\begin{IEEEeqnarray*}{rCl}
    \Pr_{x \in \binalphn}[f(x) \neq f(x^{\oplus i})] & = & \E_{x \in \binalphn}\big[\big(f(x) - f(x^{\oplus i})\big)^{2}\big] \\
                                                   & = & 4\E_{x \in \binalphn}\Bigg[\Bigg(\frac{f(x) - f(x^{\oplus i})}{2}\Bigg)^{2}\Bigg] \\
                                                   & = & 4\E[L_{i}f^{2}]
\end{IEEEeqnarray*}

\end{proof}

Por convenção, neste trabalho nós iremos usar $\E[L_{i}f^{2}]$ como a definição de $Inf_{i}[f]$ para funções $f: \binalphn \to \binalph$, ao invés de $\Pr_{x \in \binalphn}[f(x) \neq f(x^{\oplus i})]$. O mesmo vale para funções de $\binalphn$ para $\mathbb{R}$. A vantagem disso é podermos usar a fórmula que aparece em \ref{prop: derivative_inf} como a definição da influência da $i$-ésima coordenada independente se o domínio é $\binalphn$ ou $\pmonen$.

A influência total de $f$ é a soma das influências individuais de todas as coordenadas: $\I[f] = \sum_{i \in [n]} Inf_{i}[f]$. Levando em conta que $Inf_{i}[f] = \sum_{i \ni S} \widehat{f}(S)^{2}$, cada coeficiente de Fourier $\widehat{f}(S)$ aparece $\lvert S \rvert$ vezes na soma da influência total, portanto temos a seguinte fórmula: $\I[f] = \sum_{k > 0} kW^{k}[f]$.


\subsubsection{Estabilidade de ruído}

Seja $\rho \in [0, 1]$ e considere a string $y$ formada a partir de $x$ onde:

\begin{equation*}
y_{i} = \begin{cases}
            x_{i} \text{ com probabilidade } \rho \\
            \text{uniformente distribuido com probabilidade } 1 - \rho
        \end{cases}
\end{equation*}

Dizemos que $x$ e $y$ são $\rho$-correlacionados e denotamos $y \sim N_{\rho}(x)$.

\begin{defi}(Estabilidade de ruído)

Seja $f: \pmonen \to \mathbb{R}$ e $\rho \in [0, 1]$. A estabilidade de ruído de $f$ sobre $\rho$ é

\begin{equation*}
Stab_{\rho}[f] = \underset{\substack{x \sim \pmonen \\ y \sim N_{\rho}(x)}}{\E}[f(x)f(y)]
\end{equation*}

\end{defi}

Pela definição da estabilidade de ruído, nós temos o seguinte:

\begin{IEEEeqnarray*}{rCl}
    Stab_{\rho}[f] & = & \E[f(x)f(y)] \\
                   & = & \Pr[f(x) = f(y)] - \Pr[f(x) \neq f(y)] \\
                   & = & 1 - 2\Pr[f(x) \neq f(y)]
\end{IEEEeqnarray*}

Onde $x$ e $y$ são strings $\rho$-correlacionadas. Esse valor, $\Pr[f(x) \neq f(y)]$ é a \emph{sensitividade de ruído} de $f$ que denotamos por $NS_{\rho}[f]$. Ou seja, 

\begin{equation*}
NS_{\rho}[f] = \frac{1}{2} - \frac{Stab_{\rho}[f]}{2}.
\end{equation*}

Assim como fizemos para a influência também é conveniente ter uma interpretação baseada na expansão de Fourier para a estabilidade de ruído e sensitividade de ruído de uma função. Para isso consideramos o operador de ruído $T_{\rho}$ definido como

\begin{equation*}
    T_{\rho}f(x) = \E_{y \sim N_{\rho}(x)}[f(y)].
\end{equation*}

\begin{prop} \label{prop: T_fexpan}
    Seja $f: \pmonen \to \mathbb{R}$, a expansão de Fourier de $T_{\rho}f$ é dada por:

    \begin{equation*}
        T_{\rho}f = \sum_{S \subseteq [n]} \rho^{\lvert S \rvert}\widehat{f}(S)\chi_{S}
    \end{equation*}

\end{prop}

\begin{proof}

    De novo, como $T_{\rho}$ é um operador linear precisamos apenas provar para o caso $f = \chi_{S}$, $S \subseteq [n]$:

    \begin{equation*}
        T_{\rho}\chi_{S}(x) = \underset{y \sim N_{\rho}(x)}{\E}[\chi_{S}(y)] = \prod_{i \in S}\underset{y \sim N_{\rho}(x)}{\E}[y_{i}] = \prod_{i \in S}(\rho x_{i}) = \rho^{\lvert S \rvert}\chi_{S}(x)
    \end{equation*}  

    Donde nós usamos que os bits $y_{i}$ são mutualmente independentes e têm expectativa $\rho x_{i}$.  

\end{proof}

Daí nós temos que

\begin{equation*}
    Stab_{\rho}[f] = \underset{\substack{x \sim \pmonen \\ y \sim N_{\rho}(x)}}{\E}[f(x)f(y)] = \underset{x \sim \pmonen }{\E}\Big[f(x)\underset{y \sim N_{\rho}(x)}{\E}[f(y)]\Big] = \underset{x \sim \pmonen}{\E}[f(x)T_{\rho}f(x)] = \langle f, T_{\rho}f \rangle
\end{equation*}

E portanto, pelo teorema de Plancherel e proposição \ref{prop: T_fexpan}:

\begin{equation} \label{fourier_stab}
    Stab_{\rho}[f] = \langle f, T_{\rho}f \rangle = \sum_{S \subseteq [n]} \widehat{f}(S) \widehat{T_{\rho}f}(S)
                                                  = \sum_{S \subseteq [n]} \rho^{\lvert S \rvert} \widehat{f}(S)^{2}
                                                  = \sum_{k = 0}^{n} \rho^{k} W^{k}[f]
\end{equation}

\subsubsection{Hipercontratividade}

%Ainda nesta seção nós discutimos brevemente sobre hipercontratividade. Primeiro, citamos o teorema da hipercontratividade.

%\begin{teo} (Teorema da hipercontratividade)

%Seja $f \in \pmonen \to \mathbb{R}$ e sejam $1 \leq p \leq q \leq \infty$. Então $\lVert T_{\rho}f \rVert_{q} \leq \lVert f \rVert_{p}$, onde $\rho$ satisfaz $0 \leq \rho \leq \sqrt{\frac{p - 1}{q - 1}}$.

%\end{teo}

%###########################################################################Tribes_N########################################################

\begin{lema} (O Lema de Bonami) \label{bonami_lemma}

Seja $f: \{-1, 1\}^{n} \to \mathbb{R}$ tal que $\degg(f) = k$, então $\lVert f \rVert_{4} \leq \sqrt{3}^{k}\lVert f \rVert_{2}$. Equivalentemente, $\E[f^{4}] \leq 9^{k}\E[f^{2}]^{2}$.

\end{lema}

\begin{proof}

Nós provamos por indução em $n$, o número de variáveis.

Se $n = 0$ então $f$ é uma função constante e temos que $\E[f^{4}] = \E[f^{2}]^{2}$, portanto o lema é verdade para $n = 0$. 

Seja $n > 0$ e assuma que o lema é verdade para funções de $n - 1$ variáveis. Nós iremos usar o fato que $f: \{-1, 1\}^{n} \to \mathbb{R}$ pode ser escrita  como $x_{n}D_{n}f + E_{n}f$. Por conveniência nós iremos usar a mesma notação que O'Donnell: $D_{n}f = d$ e $E_{n}f = e$. Ou seja, $f = x_{n}d + e$ por esta notação. Fixe $k = \degg(f)$, temos que

\begin{IEEEeqnarray*} {rCl}
	\E[f^{4}] & = & \E \big[(x_{n}d +e)^{4} \big] \\
	                & = & \E \big[(x_{n}d)^{4} \big] + 3\E \big[(x_{n}d)^{3}e \big] + 6\E \big[(x_{n}d)^{2}e^{2} \big] + 3\E \big[(x_{n}d)e^{3} \big] + \E[e^{4}].
\end{IEEEeqnarray*}

Daí como $d$ e $e$ são independentes de $x_{n}$ e $\E[x_{n}^{4}] = \E[x_{n}^{2}] = 1$ e $\E[x_{n}^{3}] = \E[x_{n}] = 0$:

\begin{equation} \label{eq: bonami_lemma_1}
	\E[f^{4}] = \E[d^{4}] + 6\E[d^{2}e^{2}] + \E[e^{4}].
\end{equation}

E nós podemos usar a desigualdade de Cauchy-Schwarz para ``separar'' $d^{2}$ e $e^{2}$:

\begin{equation*}
	 \E[d^{4}] + 6\E[d^{2}e^{2}] + \E[e^{4}] \leq \E[d^{4}] + 6\sqrt{\E[d^{4}]}\sqrt{\E[e^{4}]} + \E[d^{4}].
\end{equation*}

E como $d$ e $e$ são funções de $n - 1$ variáveis nós podemos usar a hipótese de indução. Precisamente, pela hipótese de indução temos que $\E[d^{4}] \leq 9^{k - 1}\E[d^{2}]^{2}$ (pois $\degg(d) = k - 1$) e $\E[e^{4}] \leq 9^{k} \E[e^{2}]^{2}$ (pois $\deg(e) = k$), daí

\begin{IEEEeqnarray*} {rCl}
	\E[d^{4}] + 6\sqrt{\E[e^{4}]}\sqrt{\E[e^{4}]} + \E[e^{4}] & \leq & 9^{k - 1}\E[d^{2}]^{2} + 6\sqrt{9\E[d^{2}]^{2}}\sqrt{9\E[e^{2}]^{2}} + 9\E[e^{2}]^{2} \\
													      & \leq & 9^{k} \big(\E[d^{2}]^{2} + 2\E[d^{2}]\E[e^{2}] + \E[e^{2}]^{2} \big) \\
													      & = & 9^{k} \big(\E[d^{2}] + \E[e^{2}]\big)^{2}.
\end{IEEEeqnarray*}

Portanto $\E[f^{4}] \leq 9^{k} \big(\E[d^{2}] + \E[e^{2}]\big)^{2}$ e agora só precisamos mostrar que $\E[d^{2}] + \E[e^{2}] = \E[f^{2}]$. Nós usamos basicamente o mesmo argumento que usamos para mostar \ref{eq: bonami_lemma_1}:

\begin{IEEEeqnarray*} {rCl}
	\E[f^{2}] & = & \E[(x_{n}d + e)^{2}] \\
	                & = & \E[(x_{n}d)^{2}] + 2\E[x_{n}de] + \E[e^{2}] \\
	                & = & \E[d^{2}] + \E[e^{2}].
\end{IEEEeqnarray*}

Daí podemos finalmente concluir que $\E[f^{4}] \leq 9^{k}\E[f^{2}]^{2}$.

\end{proof}

\subsubsection{Tribes$_{N}$}

A função Tribes$_{w, s}: \binalph^{ws} \to \binalph$ é definida como sendo $\bigvee_{i = 1}^{s}\bigwedge_{j = 1}^{w} x_{i, j}$. Nós chamamos cada termo de \emph{tribo} e portanto $Tribes_{w, s} = 1$ se e somente se pelo menos uma tribo é unanimamente 1.

Essa função (até onde eu saiba) apareceu primeiro em \cite{ben1990collective}, em que Ben-Or e Linial a usaram como exemplo de uma função que é ao mesmo tempo equilibrada e que tem influência máxima pequena, mais ou menos próxima do mínimo imposto pela desigualdade de Poincaré: $\underset{i \in [n]}{max}\{Inf_{i}[f]\} \geq \frac{1}{n}$.

Fixando $w \geq 1$, nós vamos considerar a função Tribes$_{n}$ = Tribes$_{s, w}$, onde $n = sw$ e $s$ é o maior inteiro tal que $(1 - 2^{-w})^{s} \geq 1/2$. Com essas escolhas de $s$ e $w$ nós temos o seguinte.

\begin{prop} \label{prop: tribes_n}

    Seja $w \geq 1$, $s$ o maior inteiro satisfazendo $(1 - 2^{-w})^{s} \geq 1/2$ e $n = ws$, então: 

    \begin{itemize}

        \item $w = logn - loglogn - o(1)$.

        \item $s = \Theta(\frac{n}{logn})$.

    \end{itemize}

\end{prop}

\begin{proof}

    Primeiro nós achamos uma expressão para $s$. Usando a desigualdade $e^{x} \geq 1 + x$, para todo $x \in \mathbb{R}$, temos que $e^{-s2^{-w}} \geq (1 - 2^{-w})^{s} \geq 1/2$, e portanto quanto tiramos o logaritmos de ambos os lados obtemos $-s2^{-w} \geq -ln(2)$ e rearranjando:

    \begin{equation*}
        s \leq 2^{w}ln(2)
    \end{equation*}

    E também, como escolhemos $s$ de forma que $(1 - 2^{-w})^{s + 1} < 1/2$:

    \begin{equation*}
        s + 1 \geq \frac{-ln(2)}{ln(1 - 2^{-w})} \geq (2^{w} - 1)ln(2)
    \end{equation*}

    Donde nós usamos que $ln(1 - 2^{-w}) \geq \frac{1}{1 - 2^{w}}$. Juntando tudo temos que 

    \begin{equation*}
       2^{w}ln(2) - ln(2) - 1 \leq s \leq 2^{w}ln(2)
    \end{equation*} 

    Então, para algum $\alpha_{w} \in [0, 2]$ apropriado, temos que $s = 2^{w}ln(2) - \alpha_{w}$. 

    Agora, $n = ws = w2^{w}ln(2) - w\alpha_{w}$ e consequentemente $n \leq w2^{w}ln(2)$ e $n \geq w2^{w}ln(2) - 2w$, ou seja:

    \begin{equation*}
        \frac{n}{ln(2)} \leq w2^{w} \leq \frac{n}{ln(2) - 2^{-w + 1}}
    \end{equation*}

    E vemos que com $w$ tendendo ao infinito, $n/ln(2) = w2^{w}$ e portanto $w = logn - loglogn - o(1)$. Quando substituimos este valor $w$ na expressão que encontramos para $s$ temos que $s = \Theta(\frac{n}{logn})$.

\end{proof}

Da forma que definimos $n$ para um $w$ fixo nós temos na verdade uma sequência $\{n_{w}\}_{w \geq 1}$, e podemos verificar que $n_{w + 1} > 2w_{w}$.

Agora também podemos computar a influência total de Tribes$_{n}$.

\begin{prop}

    Para todo $i \in [n]$, Inf$_{i}$[Tribes$_{n}$] = $\mathcal{O}(\frac{logn}{n})$.

\end{prop}

\begin{proof}

    Usando a nossa definição para as influências individuais para funções de $\binalphn$ para $\binalph$, temos que

\begin{equation*}
Inf_{i}[\Tribes_{n}] = \E[L_{i}\Tribes_{n}^{2}] = \frac{1}{4}\Pr_{x \sim \binalphn}[\Tribes_{n}(x) \neq \Tribes_{n}(x^{\oplus i})]
\end{equation*}

    Para que uma coordenada $i$ seja pivotal para uma entrada $x \in \binalphn$ é suficiente e necessário que todas as outras coordenadas na mesma tribo que $i$ sejam 1 e que nenhuma outra tribo seja 1 unanimamente. Traduzindo:

    \begin{equation*}
        Inf_{i}[\Tribes_{n}] = \frac{1}{4}(2^{-w + 1}(1 - 2^{-w})^{s - 1}) = \frac{1}{2^{w + 1} - 2}Pr[Tribes_{n}(x) = 0]
    \end{equation*}

    Agora, seja $s^{\prime} \in \mathbb{R}$ tal que $(1 - 2^{-w})^{s^{\prime}} = 1/2$ e $\epsilon \in [0, 1)$ tal que $s^{\prime} = s + \epsilon$. Então temos:

    \begin{IEEEeqnarray*}{rCl}
        Pr[Tribes_{n}(x) = 1] & = & (1 - 2^{-w})^{s} \\
                               & = & (1 - 2^{-w})^{s^{\prime}}(1 - 2^{-w})^{-\epsilon} \\
                               & = & \frac{1}{2}(1 + \epsilon 2^{-w} + \mathcal{O}(2^{-2w})) \\
    \end{IEEEeqnarray*}

    E como $w = logn - loglogn - o(1)$,

    \begin{equation*}
        Pr[Tribes_{n}(x) = 1] = \frac{1}{2} + \mathcal{O}(\frac{logn}{n})
    \end{equation*}
    
    E como $\frac{1}{2^{w + 1} - 2} = \mathcal{O}(\frac{logn}{n})$ o resultado segue.

\end{proof}

Para cada $i \in [s]$ denotaremos por $T_{i}$ o conjunto de índices das variáveis na $i$-ésima tribo e definimos $f_{i}$ como

\begin{equation*}
    f_{i}(x) = \begin{cases}
                    1 \text{ se a } i\text{-ésima tribo não é unanimamente 1 sobre a entrada } x \\
                    0 \text{ caso contrário}
               \end{cases}
\end{equation*}

Desta forma Tribes$_{n}(x) = 1 - \prod_{i = 1}^{s} f_{i}(x)$ e usamos esta forma de representar Tribes$_{n}$ para calcular seus coeficientes de Fourier.

\begin{prop} \label{prop: tribes_fourier_coef}

    Seja $S \subseteq [n]$, $(S_{1}, S_{2}, \dots, S_{s})$ uma partição de $S$ tal que $S_{i} = S \cap T_{i}$, para cada $i \in [s]$, e k = $\#\{i \rvert S_{i} \neq \emptyset\}$ então

    \begin{equation*}
    \widehat{Tribes_{n}}(S) = \begin{cases}
                                   1 - (1 - 2^{-w})^{s} \text{ se } S = \emptyset \\
                                   (-1)^{\lvert S \rvert + k + 1}2^{-kw}(1 - 2^{-w})^{s - k} \text{ se } S \neq \emptyset
                               \end{cases}
    \end{equation*}

\end{prop}

\begin{proof}

    O caso $S = \emptyset$ é verdade pois

    \begin{equation*}
        \widehat{\Tribes_{n}}(\emptyset) = \E_{x \sim \binalphn}[\Tribes_{n}(x)] = \Pr_{x \sim \binalphn}[\Tribes_{n}(x) = 1] = 1 - (1 - 2^{-w})^{s}
    \end{equation*}

    Para o caso geral nós temos

    \begin{IEEEeqnarray*}{rCl}
        \widehat{Tribes_{n}}(S) & = & \underset{x \sim \binalphn}{\E}[\Tribes_{n}(x)\chi_{S}(x)] \\
                                & = & \underset{x \sim \binalphn}{\E}\big[(1 - \prod_{i = 1}^{n}f_{i}(x))\chi_{S}(x)\big] \\
                                & = & -\prod_{i = 1}^{s} \underset{x \sim \binalphn}{\E}[f_{i}(x)\chi_{S_{i}}(x)] \\
                                & = & -\prod_{i = 1}^{s} \widehat{f_{i}}(S_{i})
    \end{IEEEeqnarray*}

    Então só precisamos analizar $\widehat{f_{i}}(S_{i})$.

    \begin{IEEEeqnarray*}{rCl}
        \widehat{f_{i}}(S_{i}) & = & 2^{-n} \sum_{x \in \binalphn} f_{i}(x)\chi_{S_{i}}(x) \\
                               & = & 2^{-n} \sum_{\substack{x \in \binalphn \\ x_{j} = 1 \text{ para todo } j \in T_{i}}} (-1)^{\lvert S_{i} \rvert}f_{i}(x) + 2^{-n} \sum_{\substack{x \in \binalphn \\ x_{j} \neq 1 \text{ para algum } j \in T_{i}}}\chi_{S_{i}}(x)
    \end{IEEEeqnarray*}

    Cada termo da primeira soma é 1 sempre que pelo menos uma das $w - \lvert S_{i} \rvert$ variáveis em $T_{i} \setminus S_{i}$ é 1, o que acontece com probabilidade $(1 - 2^{-w + \lvert S_{i} \rvert})$. A segunda soma pode ser decomposta pela quantidade de variáveis com índices em $S_{i}$ que são 0:

    \begin{IEEEeqnarray*}{rCl}
        \widehat{f_{i}}(S_{i}) & = & (-1)^{\lvert S_{i} \rvert}2^{-\lvert S_{i} \rvert}(1 - 2^{-w + \lvert S_{i} \rvert}) + 2^{-n}\sum_{k = 1}^{\lvert S_{i} \rvert}(-1)^{\lvert S_{i} \rvert - k}\binom{\lvert S_{i} \rvert}{k}2^{n - \lvert S_{i} \rvert} \\
                               & = & (-1)^{\lvert S_{i} \rvert + 1}2^{-w} + (-1/2)^{\lvert S_{i} \rvert}\sum_{k = 0}^{\lvert S_{i} \rvert}(-1)^{k}\binom{\lvert S_{i} \rvert}{k} \\
                               & = & (-1)^{\lvert S_{i} \rvert + 1}2^{-w}
    \end{IEEEeqnarray*}
    Lembrando que $k = \#\{i \lvert S_{i} \neq \emptyset\}$, podemos concluir:

    \begin{IEEEeqnarray*}{rCl}
        \widehat{Tribes_{n}}(S) & = & -\prod_{i = 1}^{s}\widehat{f_{i}}(S_{i}) \\
                                & = & -\Bigg(\prod_{i : S_{i} \neq \emptyset}(-1)^{\lvert S_{i} \rvert + 1}2^{-w}\Bigg)(1 - 2^{-w})^{s - k} \\
                                & = & (-1)^{\lvert S \rvert + k + 1}2^{-kw}(1 - 2^{-w})^{s - k}
    \end{IEEEeqnarray*}

\end{proof}

\section{Algumas aplicações}

Agora nós iremos ver algumas aplicações da análise de funções Booleanas.

\subsection{Teste de propriedade}

Uma propriedade de funções é um conjunto de $\mathcal{P}_{n}$ de funções $\{-1, 1\}^{n} \to \{-1, 1\}$. Dado acesso à uma função $f: \{-1, 1\}^{n} \to \{-1, 1\}$, nós queremos um algoritmo que decida se $f$ satisfaz a propriedade $\mathcal{P}_{n}$, e também queremos que o nosso algoritmo seja robusto de forma que todas as funções $f$ que fazem o algoritmo aceitar com alta probabilidade estejam próximas de terem a propriedade $\mathcal{P}_{n}$ (no sentido que precisamos apenas inverter o valor de $f$ em uma fração pequena das entradas para obter uma função que tem a propriedade $\mathcal{P}_{n}$).

\begin{defi} \label{defi: property_test}

Seja $q \geq 1$, $\lambda > 0$, $\varepsilon > 0$, $\mu$ uma distribuição qualquer de strings em $\{-1, 1\}^{n}$ e $\mathcal{P}_{n}$ um subconjunto de funções $\{-1, 1\}^{n} \to \{-1, 1\}$. Um algoritmo $A$ que testa a propriedade $\mathcal{P}_{n}$  que faz $q$ consultas é um algoritmo que ao ter acesso à uma função $f: \{-1, 1\}^{n} \to \{-1, 1\}$, tira strings $x_{1}, \dots, x_{q}$ da distribuição $\mu$, consulta os valores $f(x_{1}), \dots, f(x_{q})$ e decide se aceita ou rejeita $f$ de forma que:

\begin{itemize}

	\item Completude: se $f \in \mathcal{P}_{n}$ então $A$ aceita com probabilidade 1.
	
	\item Robustez: se $A$ aceita $f$ com probabilidade maior do que $1 - \lambda\varepsilon$, então $f$ é $\varepsilon$-próxima de alguma função em $\mathcal{P}_{n}$.

\end{itemize}

\end{defi} 

Agora nós queremos mostrar que funções lineares são testáveis com apenas 3 consultas. Antes disto, precisamos fazer algumas observações que irão facilitar a prova.

\begin{defi} \label{convolution}

Sejam $f, g: \mathbb{F}_{2}^{n} \to \mathbb{R}$. A convolução de $f$ e $g$ é definida como

\begin{equation*}
	(f*g)(x) = \E_{y \sim \mathbb{F}_{2}^{n}}[f(y)g(x + y)].
\end{equation*}

\end{defi}

Para a prova da proposição a seguir precisaremos do fato que se $x, y \in \mathbb{F}_{2}^{n}$ e $S \subseteq [n]$, então $\chi_{S}(x)\chi_{S}(y) = \chi_{S}(x + y)$.

\begin{prop} \label{convolution_fourier}

Sejam $f, g: \mathbb{F}_{2}^{n} \to \mathbb{R}$, então para todo $S \subseteq [n]$ temos que $\widehat{(f*g)}(S) = \widehat{f}(S)\widehat{g}(S)$.

\end{prop}

\begin{proof}

Usando a definição dos coeficientes de Fourier e o fato que $x$ e $y$ são independentes (e portanto $x + y$ é distribuido uniformemente),

\begin{IEEEeqnarray*} {rCl}
	\widehat{(f*g)}(S) & = & \E_{x \sim \mathbb{F}_{2}^{n}}[(f*g)(x)\chi_{S}(x)] \\
			          & = & \E_{x \sim \mathbb{F}_{2}^{n}}\Big[\E_{y \sim \mathbb{F}_{2}^{n}}[f(y)g(x + y)]\chi_{S}(x)\Big] \\
			          & = & \E_{x, y \sim \mathbb{F}_{2}^{n}}[f(y)g(x + y)\chi_{S}(x)] \\
			          & = & \E_{z, y \sim \mathbb{F}_{2}^{n}}[f(y)g(z)\chi_{S}(y + z)] \\
			          & = & \E_{y \sim \mathbb{F}_{2}^{n}}[f(y)\chi_{S}(y)]\E_{z \sim \mathbb{F}_{2}^{n}}[g(z)\chi_{S}(z)] \\
			          & = & \widehat{f}(S)\widehat{g}(S).
\end{IEEEeqnarray*}

\end{proof}

Nós dizemos que uma função $f: \mathbb{F}_{2}^{n} \to \mathbb{F}_{2}$ é linear se e somente se existe um $S \subseteq [n]$ tal que $f = \sum_{i \in S}x_{i}$. Seguindo essa definição, temos que se $f$ for linear então para todos $x, y \in \mathbb{F}_{2}^{n}$ é verdade que $f(x) + f(y) = f(x + y)$. Na verdade, se definissemos uma função linear como todas funções $f: \mathbb{F}_{2}^{n} \to \mathbb{F}_{2}$ tal que para todos pares $x, y$ é verdade que $f(x) + f(y) = f(x + y)$, nós teriamos uma definição equivalente de uma função linear. Para facilitar nossa vida nós iremos considerar funções lineares com contradomínio $\{-1, 1\}$. Primeiro nós demonstramos a seguinte proposição.

\begin{prop} \label{linear_prox}

Sejam $f: \mathbb{F}_{2}^{n} \to \{-1, 1\}$ e $\varepsilon > 0$. Se $\Pr_{x, y \sim \mathbb{F}_{2}^{n}}[f(x)f(x) = f(x + y)] > 1- \varepsilon$ então $f$ está $\varepsilon$-próxima de uma função linear.

\end{prop}

\begin{proof}

Seja $S^{*}$ tal que $\widehat{f}(S^{*})$ é máximo. Ou seja, para todo $S \subseteq [n]$ é verdade que $\widehat{f}(S^{*}) \geq \widehat{f}(S)$. Nós mostraremos que $f$ está $\varepsilon$-próxima de $\chi_{S^{*}}$. Pela forma como definimos $S^{*}$ e o teorema de Parseval temos

\begin{equation*}
	\dist(f, \chi_{S^{*}}) = \frac{1}{2} - \frac{1}{2}\widehat{f}(S^{*}) = \frac{1}{2} - \frac{1}{2}\Big(\widehat{f}(S^{*})\sum_{S \subseteq [n]}\widehat{f}(S)^{2} \Big) \leq \frac{1}{2} - \frac{1}{2}\sum_{S \subseteq [n]}\widehat{f}(S)^{3}.
\end{equation*}

Pelo teorema de Plancherel e a proposição \ref{convolution_fourier} nós sabemos que

\begin{equation*}
	\sum_{S \subseteq [n]}\widehat{f}(S)^{3} = \langle f, f*f \rangle = \E_{x \sim \mathbb{F}_{2}^{n}}[f(x)(f*f)(x)]
\end{equation*}

e portanto,

\begin{IEEEeqnarray*} {rCl}
	\dist(f, \chi_{S^{*}}) & \leq &  \frac{1}{2} + \frac{1}{2}\E_{x \sim \mathbb{F}_{2}^{n}}[f(x)f*f(x)] \\
	                                   & = & \frac{1}{2} + \frac{1}{2}\E_{x \sim \mathbb{F}_{2}^{n}}\Big[f(x) \E_{y \sim \mathbb{F}_{2}^{n}}[f(y)f(x + y)]\Big] \\
	                                   & = & \frac{1}{2} + \frac{1}{2}\E_{x, y \sim \mathbb{F}_{2}^{n}}[f(x)f(y)f(x + y)]
\end{IEEEeqnarray*}

Daí, como estamos assumindo que $\Pr_{x, y \sim \mathbb{F}_{2}^{n}}[f(x)f(y) \neq f(x + y)] < \varepsilon$:

\begin{IEEEeqnarray*} {rCl}
	\frac{1}{2} + \frac{1}{2}\E_{x, y \sim \mathbb{F}_{2}^{n}}[f(x)f(y)f(x + y)] & = & \frac{1}{2} + \frac{1}{2}(2\Pr_{x, y \sim \mathbb{F}_{2}^{n}}[f(x)f(y) = f(x + y)] - 1) \\ 
	                                                                                                                               & < & \frac{1}{2} + \frac{1}{2}(2\varepsilon - 1) = \varepsilon.
\end{IEEEeqnarray*}

Então concluimos que $\dist(f, \chi_{S^{*}}) < \varepsilon$.

\end{proof}

Agora nós podemos ver que testar se $f$ é uma função linear pode ser feito com apenas 3 consultas.

\begin{teo} (Blum-Luby-Rubinfeld) \label{BLR_test}

Os seguintes teste é um teste de propriedade para funções lineares:

\begin{itemize}

	\item Escolha $x, y \sim \mathbb{F}_{2}^{n}$.
	
	\item Consulta $f$ nas strings $x, y$ e $x + y$.
	
	\item Aceita se e somente se $f(x)f(y) = f(x + y)$.

\end{itemize}

\end{teo}

\begin{proof}

Por definição se $f$ for linear então o algoritmo irá aceitar com probabilidade 1. Além do mais, se o teste aceita uma função $f$ com probabilidade maior do que $1 - \varepsilon$ temos pela proposição \ref{linear_prox} que $f$ está $\varepsilon$-próxima de uma função linear.

\end{proof}

Além de testar se $f: \mathbb{F}_{2}^{n} \to \{-1, 1\}$ está próxima de uma das funções paridade $\chi_{S}$, também podemos ``recuperar'' $\chi_{S}$ a partir de $f$.

\begin{teo} \label{local_correcting_routine}

Sejam $\varepsilon > 0$ e $f: \mathbb{F}_{2}^{n} \to \mathbb{F}_{2}$ uma função $\varepsilon$-próxima de uma função paridade $\chi_{S}$, para algum $S \subseteq [n]$, seja $x \in \mathbb{F}_{2}^{n}$ e considere o seguinte algoritmo: 

\begin{itemize}

	\item Escolha $y \sim \mathbb{F}_{2}^{n}$.
	
	\item Dê como saída $f(y) + f(x + y)$.

\end{itemize}

Então, com probabilidade maior do que $1 - 2\varepsilon$ o algoritmo acima dá como saída o valor $\chi_{S}(x)$.

\end{teo}

\begin{proof}

Como $y$ e $x + y$ são uniformes (mesmo que não sejam independentes) temos que, com probabilidade maior do que $1 - 2\varepsilon$, $f(y) = \chi_{S}(y)$ e $f(x + y) = \chi_{S}(x + y)$, e portanto

\begin{equation*}
	f(y)f(x + y) = \chi_{S}(y)\chi_{S}(x + y) = \chi_{S}(y + (x + y)) = \chi_{S}(x).
\end{equation*}

\end{proof}

\subsubsection{Testando simetria}

Agora nós mostramos como testar se uma dada função é simétrica.

Primeiro nós fazemos a seguinte definição.

\begin{defi}

Seja $f: \mathbb{F}_{2}^{n} \to \{-1, 1\}$, então o operador $S$ age sobre $f$ da seguinte forma:

\begin{equation*}
	Sf(x) = \E_{\pi \sim S_{n}}[f(\pi(x))].
\end{equation*}

\end{defi}

De certa forma, o operador $S$ transforma $f$ em uma função simétrica. Se $f$ for simétrica, é fácil ver que $Sf = f$. Além do mais, se $f$ estiver ``próxima de ser simétrica'', então $Sf$ e $f$ também estão próxima.

\begin{prop} \label{f_sf_1_norm}

Sejam $f: \mathbb{F}_{2}^{n} \to \{-1, 1\}$ e $\varepsilon > 0$. Suponha que $\Pr_{x, \pi}[f(x) = f(\pi(x))] > 1 - \varepsilon$ então

\begin{equation*}
	\lVert f - Sf \rVert_{1} < 2\varepsilon.
\end{equation*}

\end{prop}

\begin{proof}

Nós temos que

\begin{equation*}
	\varepsilon > \Pr_{x, \pi}[f(x) \neq f(\pi(x))] = \E_{x, \pi}[\mathbbm{1}_{f(x) \neq f(\pi(x))}(x, \pi)]
\end{equation*}

Sempre que $f(x) \neq f(\pi(x))$ é verdade que $\lvert f(x) - f(\pi(x)) \rvert = 2 = 2\times \mathbbm{1}_{f(x) \neq f(\pi(x))}(x, \pi)$ e portanto,

\begin{IEEEeqnarray*} {rCl}
	\E_{x, \pi}[\mathbbm{1}_{f(x) \neq f(\pi(x))}(x, \pi)] & = & \frac{1}{2}\E_{x, \pi}\big[\lvert f(x) - f(\pi(x)) \rvert \big] \\
	                                                                                      & \geq & \frac{1}{2}\E_{x \sim \mathbb{F}_{2}^{n}}\Big[ \big\lvert f(x) - \E_{\pi \sim S_{n}}[f(\pi(x))] \big\rvert \Big] \\
	                                                                                      & = & \frac{1}{2}\E_{x \sim \mathbb{F}_{2}^{n}} \big[\lvert f(x) - Sf(x) \rvert \big] \\
	                                                                                      & = & \frac{1}{2} \lVert f - Sf \rVert_{1}.
\end{IEEEeqnarray*}

Daí nós temos que $\lVert f - Sf \rVert_{1} < 2\varepsilon$.

\end{proof}

\begin{lema} \label{sign_lemma}

Sejam $f: \mathbb{F}_{2}^{n} \to \{-1, 1\}$ e $g: \mathbb{F}_{2}^{n} \to \mathbb{R}$. Defina $h_{\theta}$, para cada $\theta \in [-1, 1]$, como $h_{\theta}(x) = \sgn(g(x) - \theta)$. Se $\lVert f - g \rVert_{1} < \varepsilon$, então

\begin{equation*}
	\E_{\theta \sim [-1, 1]}[\dist(f, h_{\theta})] < \varepsilon/2.
\end{equation*}

\end{lema}

\begin{proof}

Nós temos que

\begin{IEEEeqnarray*} {rCl}
	\E_{\theta \sim [-1, 1]}[\dist(f, h_{\theta})] & = & \E_{\theta \sim [-1, 1]} \Big[\Pr_{x \sim \mathbb{F}_{2}^{n}}[f(x) \neq h_{\theta}(x)] \Big] \\ 
	                                                                        & = & \E_{\theta \sim [-1, 1]} \Big[ \E_{x \sim \mathbb{F}_{2}^{n}}[\mathbbm{1}_{f(x) \neq \sgn(g(x) - \theta)}(x, \theta)] \Big] \\
	                                                                        & = & \E_{x \sim \mathbb{F}_{2}^{n}} \Big[ \E_{\theta \sim [-1, 1]}[\mathbbm{1}_{f(x) \neq \sgn(g(x) - \theta)}(x, \theta)] \Big]
\end{IEEEeqnarray*}

Então nós vamos ver que $\E_{\theta \sim [-1, 1]}[\mathbbm{1}_{f(x) \neq \sgn(g(x) - \theta)}(x, \theta)] \leq \frac{1}{2}\lvert f(x) - g(x) \rvert$. O caso em que $\lvert f(x) - g(x) \rvert \geq 2$ é trivial.  Também não vamos nos preocupar com o caso em que $\lvert g(x) \rvert \geq \lvert f(x) \rvert$, pois $\E_{\theta \sim [-1, 1]} [\mathbbm{1}_{f(x) \neq \sgn(g(x) - \theta)}(x, \theta)] = 0$. Suponha então que $\lvert f(x) - g(x) \rvert < 2$ e que $\lvert g(x) \rvert < \lvert f(x) \rvert$. Sem perda de generalidade vamos supor que $f(x) = - 1$ (o caso $f(x) = 1$ é simétrico). Note que quando nós tiramos $\theta$ de $[-1, 1]$ uniformemente, nós estamos basicamente tirando, uniformemente, um ponto em $[g(x) - 1, g(x) + 1]$, e portanto,

\begin{IEEEeqnarray*} {rCl}
	\Pr_{\theta \sim [-1, 1]}[f(x) \neq \sgn(g(x) - \theta)] & = & \Pr_{\theta \sim [-1, 1]}[g(x) - \theta \geq 0] \\
									  & = & \Pr_{\theta \sim [-1, 1]}[g(x) \geq \theta] \\
									  & = & \Pr_{\theta \sim [-1, 1]}[\theta \in [-1, g(x)]] \\
									  & = & \frac{1}{2}\lvert -1 - g(x) \rvert \\
									  & = & \frac{1}{2}\lvert f(x) - g(x) \rvert.
\end{IEEEeqnarray*}

Então temos que

\begin{equation*}
	\E_{x \sim \mathbb{F}_{2}^{n}} \Big[ \E_{\theta \sim [-1, 1]}[\mathbbm{1}_{f(x) \neq \sgn(g(x) - \theta)}(x, \theta)] \Big] \leq \frac{1}{2}\E_{x \sim \mathbb{F}_{2}^{n}}[\lvert f(x) - g(x) \lvert] = \frac{1}{2} \lVert f - g \rVert_{1}.
\end{equation*}

Daí, como $\lVert f - g \rVert_{1} < \varepsilon$, temos que $\E_{\theta \sim [-1, 1]}[\dist(f, h_{\theta})] < \varepsilon/2$.

\end{proof}

\begin{prop} \label{close_to_symmetric}

Seja $f: \mathbb{F}_{2}^{n} \to \{-1, 1\}$ e suponha que $\Pr_{x, \pi}[f(x) = f(\pi(x))] > 1 - \varepsilon$, então $f$ está $\varepsilon$-próxima de uma função simétrica $g: \mathbb{F}_{2}^{n} \to \{-1, 1\}$.

\end{prop}

\begin{proof}

Como vimos em \ref{f_sf_1_norm}, $\Pr_{x, \pi}[f(x) = f(\pi(x))] > 1 - \varepsilon$ implica em $\lVert f - Sf \rVert_{1} < 2\varepsilon$, que por sua vez implica em $\E_{\theta \sim [-1, 1]}[\dist(f, h_{\theta})] < \varepsilon$, onde $h_{\theta} = \sgn(Sf - \theta)$, por causa do lema \ref{sign_lemma}. Então, em particular, deve existir um $\theta \in [-1, 1]$ tal que $\dist(f, h_{\theta}) < \varepsilon$. Para qualquer $x \in \mathbb{F}_{2}^{n}$ e $\pi \in S_{n}$, e lembrando que $Sf$ é uma função simétrica:

\begin{equation*}
	h_{\theta}(\pi(x)) = \sgn(Sf(\pi(x)) - \theta) = \sgn(Sf(x) - \theta) = h_{\theta}(x)
\end{equation*}

E portanto $h_{\theta}$ é uma função simétrica. Fazendo $g = h_{\theta}$ no enunciado da proposição obtemos o resultado.

\end{proof}

Agora temos tudo que precisamos para demonstrar um teste de simetria.

\begin{teo}

Seja $P = \{P_{n}\}_{n \geq 1} = \bigcup_{n \geq 1}\{ f: \mathbb{F}_{2}^{n} \to \mathbb{F}_{2} \lvert f \text{ é simétrica} \}$, então o seguinte algoritmo é um teste de propriedade para $P$:

\begin{itemize}

	\item Escolha $x \sim \mathbb{F}_{2}^{n}$ e $\pi \sim S_{n}$ uniformemente e independentemente.
	
	\item Consulte os valores de $f(x)$ e $f(\pi(x))$.
	
	\item Aceite se e somente se $f(x) = f(\pi(x))$.

\end{itemize}

\end{teo}

\begin{proof}

Se $f$ é uma função simétrica então claramente o algoritmo aceita com probabilidade 1. Agora, se o algoritmo aceita $f$ com probabilidade maior do que $1 - \varepsilon$ então pela proposição \ref{close_to_symmetric} temos que $f$ está $\varepsilon$-próxima de uma função simétrica.

\end{proof}

\subsubsection{Testando 1-juntas}

Agora nós iremos ver como testar a seguinte propriedade de funções $\{-1, 1\}^{n} \to \{-1, 1\}$:

\begin{equation*}
	\mathcal{D} = \{f: \{-1, 1\}^{n} \to \{-1, 1\} \lvert f(x) = x_{i}, \text{ para algum } i \in [n]\}.
\end{equation*}

Vamos considerar uma tripla de strings em $x, y$ e $z \in \{-1, 1\}^{n}$ tiradas da seguinte distribuição.

\begin{itemize}

	\item $x \sim \{-1, 1\}^{n}$.
	
	\item $y \sim N_{-1/3}(x)$ (ou seja, $x$ e $y$ são $(-1/3)$-correlacionadas).
	
	\item $z$ é tal que para qualquer $i \in [n]$:
	
	\begin{equation*}
		z_{i} = \begin{cases}
				\text{tirado uniformemente de } \{-1, 1\} \text{ se } x_{i} = y_{i} \\
				-x_{i} \text{ } (= - y_{i}) \text{ caso contrário}
			  \end{cases}
	\end{equation*}

\end{itemize}

Por definição nós temos que $\E[x_{i}y_{i}] = -1/3$, e como $x$ e $y$ são strings uniformemente aleatórias do ponto de vista de $z$, temos que

\begin{equation*}
	\E[x_{i}z_{i}] = \E[y_{i}z_{i}] = (1/3)\E[y_{i}z_{i} \lvert x_{i} = y_{i}] + (2/3)\E[y_{i}z_{i} \lvert x_{i} \neq y_{i}]=-1/3 + 2/3 \times 0 = -1/3.
\end{equation*}

Em outras palavras, os pares $(y, z)$ e $(x, z)$ também são $(-1/3)$-correlacionados. Em \emph{outras} palavras, $\E[f(x)f(y)] = \E[f(y)f(z)] = \E[f(x)f(z)] = \Stab_{-1/3}[f]$. A idéia por trás de tirar $(x, y, z)$ da distribuição acima é que cada tripla $(x_{i}, y_{i}, z_{i})$ é tirada uniformemente de $\{-1, 1\}^{3} \setminus \big((-1, -1, -1) \cup (1, 1, 1) \big)$. Seja $\NAE: \{-1, 1\}^{3} \to \{0, 1\}$ definida da seguinte forma:

\begin{equation*}
	\NAE(a, b, c) = \begin{cases}
	 			    	0 & \text{se a = b = c} \\
	 			    	1 & \text{caso contrário}
			    \end{cases}
\end{equation*}

Daí, se $x, y$ e $z$ são tiradas da distribuição acima e $f: \{-1, 1\}^{n} \to \{-1, 1\}$ for qualquer uma das $n$ 1-juntas então $\NAE(f(x), f(y), f(z)) = 1$.

A versão algebratizada de $\NAE$ é $\NAE(a, b, c) = \frac{3}{4} - \frac{1}{4}ab - \frac{1}{4}bc - \frac{1}{4}ac$ e portanto, se $x, y$ e $z$ forem tiradas da distribuição $\mathcal{D}$,

\begin{IEEEeqnarray*} {rCl}
	\E[\NAE(f(x), f(y), f(z))] & = & \frac{3}{4} - \frac{1}{4}\E[f(x)f(y)] - \frac{1}{4}\E[f(y)f(z)] - \frac{1}{4}\E[f(x)f(z)] \\
	                                        & = & \frac{3}{4} - \frac{3}{4}\E[f(x)f(y)] \\
	                                        & = & \frac{3}{4} - \frac{3}{4}\Stab_{-1/3}[f].
\end{IEEEeqnarray*}

Lembrando a expressão para $\Stab_{\rho}[f]$ usando a expansão de Fourier de $f$ \ref{fourier_stab},

\begin{IEEEeqnarray*} {rCl}
	\frac{3}{4} - \frac{3}{4}\Stab_{-1/3}[f] & = & \frac{3}{4} - \frac{3}{4}\sum_{k = 0}^{n}(-1/3)^{k}W^{k}[f] \\
						     	  & \leq & \frac{3}{4} - \frac{3}{4}((-1/3)W^{1}[f] + (-1/3)^{3}W^{3}[f] + (-1/3)^{5}W^{5}[f] + \dots) \\
						     	  & \leq & \frac{3}{4} + \frac{1}{4}W^{1}[f] + \frac{1}{36}(W^{3}[f] + W^{5}[f] + \dots) \\
						     	  & \leq & \frac{3}{4} + \frac{1}{4}W^{1}[f] + \frac{1}{36}(1 - W^{1}[f]) \\
						     	  & = & \frac{7}{9} + \frac{2}{9}W^{1}[f].
\end{IEEEeqnarray*}

Se $\E[\NAE(f(x), f(y), f(z))] = 1$ então $W^{1}[f] \geq 1$, o que significa que $f$ é uma 1-junta. Este resultado é uma prova alternativa do teorema de Arrow. Também temos que se $\E[\NAE(f(x), f(y), f(z))] > 1 - \varepsilon$, então $W^{1}[f] > 1 - \mathcal{O}(\varepsilon)$. Um teorema de Friedgut, Kalai e Naor diz que se $W^{1}[f] > 1 - \varepsilon$ então $f$ está $\mathcal{O}(\varepsilon)$-próxima de uma 1-junta ou a tua negação.

Então temos quase tudo para elaborar um teste para 1-juntas, porém, como foi dito, apenas dizer que $W^{1}[f] > 1 - \varepsilon$ não implica necessariamente em $f$ estar próxima de uma 1-junta porque o teorema de FKN diz que $f$ pode estar próxima de uma 1-junta ou de uma negação de uma 1-junta, mas queremos testar se $f = \chi_{i}$, para algum $i \in [n]$ (ou seja, não aceitamos a negação de uma 1-junta, tal funções devem ser rejeitadas com probabilidade 1). Para isso nós temos que certificar que $f$ é uma 1-junta e que $f$ é também uma função linear. A estratégia então é usar dois testes. Um deles é o teste de linearidade de BLR. O segundo teste nós apresentamos a seguir.

\begin{NAEtest} \label{NAEtest}

Dado acesso à uma função $f: \{-1, 1\}^{n} \to \{-1, 1\}$, o teste $\NAE$ faz o seguinte:

\begin{itemize}

	\item Tira $x, $y e $z$ da distribuição $\mathcal{D}_{\NAE}$.
	
	\item Consulta os valores de f(x), f(y) e f(z).
	
	\item Aceita se e somente se $\NAE(f(x), f(y), f(z)) = 1$.

\end{itemize}

\end{NAEtest}

\begin{prop}

O teste $\NAE$ é um teste para a propriedade $P$ onde

\begin{equation*}
	P = \{f: \{-1, 1\}^{n} \to \{-1, 1\} \lvert f = \pm \chi_{i} \text{, para algum } i \in [n] \}.
\end{equation*}

\end{prop}

\begin{proof}

Se $f = \pm \chi_{i}$, para algum $i \in [n]$ então como $x, y$ e $z$ satisfazem $\NAE(x_{i}, y_{i}, z_{i}) = 1$ devemos também ter $\NAE(f(x), f(y), f(z)) = 1$, e portanto o teste $\NAE$ satisfaz a propriedade da completude. Se o teste $\NAE$ aceita $f$ com probabilidade maior do que $1 - \frac{1}{10}\varepsilon$ então temos que

\begin{equation*}
	1 -\frac{1}{10}\varepsilon < \Pr_{x, y, z \sim \mathcal{D}_{\NAE}}[\NAE(f(x), f(y), f(z) = 1)] = \E_{x, y, z \sim \mathcal{D}_{\NAE}}[\NAE(f(x), f(y), f(z))] < \frac{7}{9} + \frac{2}{9}W^{1}[f].
\end{equation*}

Portanto,

\begin{equation*}
	W^{1}[f] > 1 - 0,45\varepsilon.
\end{equation*}

E daí, pelo teorema de FKN, $f$ está $\mathcal{O}(\varepsilon)$-próxima de $\pm \chi_{i}$, para algum $i \in [n]$.

\end{proof}

Agora juntando os testes $\NAE$ e $\BLR$ nós obtemos um teste para 1-juntas.

\begin{NAEplusBLRtest}

Dado acesso à uma função $f: \{-1, 1\}^{n} \to \{-1, 1\}$, o teste $\NAEplusBLR$ faz o seguinte:

\begin{itemize}

	\item Com probabilidade 1/2 faz o teste $\BLR$ com acesso à $f$.
	
	\item Com probabilidade 1/2 faz o teste $\NAE$ com acesso à $f$.
	
	\item Aceita se e somente se o teste feito aceitou $f$.

\end{itemize}

\end{NAEplusBLRtest}

\begin{prop}

O teste $\NAEplusBLR$ é um teste para a propriedade $P$ onde

\begin{equation*}
	\{f: \{-1, 1\}^{n} \to \{-1, 1\} \lvert f = \chi_{i} \text{, para algum } i \in [n]\}.
\end{equation*}

\end{prop}

\begin{proof}

Se $f = \chi_{i}$ então já sabemos que tanto o teste $\NAE$ quanto o teste $\BLR$ aceita $f$ com probabilidade 1, e portanto $\NAEplusBLR$ também aceita $f$ com probabilidade 1. Se $\NAEplusBLR$ aceita $f$ com probabilidade maior do que $1 - \frac{1}{20}\varepsilon$ então ambos os teste devem aceitar com probabilidade maior do que $1 - \frac{1}{10}\varepsilon$. Isso quer dizer que

\begin{enumerate}

	\item Pelo teste $\NAE$: $W^{1}[f] > 1 - 0,45\varepsilon$.
	
	\item Pelo teste $\BLR$: $f$ está $\frac{1}{10}\varepsilon$-próxima de uma função linear $\chi_{S^{*}}$, para algum $S^{*} \subseteq [n]$.

\end{enumerate}

O item (2) nos diz que $\widehat{f}(S^{*}) > 1 - \frac{1}{5}\varepsilon$. Se $\lvert S^{*} \rvert > 1$ temos que

\begin{equation*}
	1 = \sum_{k = 0}^{n}W^{k}[f] \geq (1 - 0,45\varepsilon) + (1 - \frac{1}{5}\varepsilon)^{2} \geq 1 - 0,45\varepsilon + 1 -\frac{2}{5}\varepsilon = 2 - 0,85\varepsilon > 1.
\end{equation*}

Logo temos que $1 > 1$, o que é uma contradição. Portanto $\lvert S^{*} \rvert = 1$ e $f$ está $\frac{1}{10}\varepsilon$-próxima de $\chi_{i}$, onde $i$ é tal que $S^{*} = \{i\}$.

\end{proof}

Nós podemos até mesmo, dado um conjunto $S \subseteq [n]$, testar se $f = \chi_{i}$, para algum $i \in S$.

\begin{dicttest} \label{dict_subset_test}

Seja $S \subseteq[n]$ e $\mathbbm{1}_{S} \in \{-1, 1\}^{n}$ a ``string característica'' de $S$. Dado acesso à uma função $f: \{-1, 1\}^{N} \to \{-1, 1\}$:

\begin{itemize}

	\item Com probabilidade $1/2$ faz o teste $\NAEplusBLR$ com acesso à $f$.
	
	\item Com probabilidade $1/2$ faz o método de correção local para $f$ com a string $\mathbbm{1}_{S}$ e dizemos que este passo aceita se o valor de retorno é -1.
	
	\item Aceita se e somente se o teste feito aceitou.

\end{itemize}

\end{dicttest}

\begin{teo}

Seja $S$ qualquer subconjunto de $[n]$, então o teste para subconjuntos das funções ditadores que acabamos de ver é um teste para a seguinte propriedade:

\begin{equation*}
	\{f: \{-1, 1\}^{n} \to \{-1 1\} \lvert f = \chi_{i} \text{ para algum } i \in S\}
\end{equation*}

\end{teo}

\begin{proof}

Se $f = \chi_{i}$ para algum $i \in S$ então em particular $f$ é uma função ditador e portanto o teste $\NAEplusBLR$ irá aceitá-la com probabilidade 1. Além disso, o método de correção local do teorema \ref{local_correcting_routine} irá retorna $-1$ com probabilidade 1 e portanto $f$ é aceita com probabilidade 1. Agora assuma que $f$ é aceita com probabilidade maior do que $1 - \lambda\varepsilon/10$. Então o teste $\NAEplusBLR$ deve necessariamente aceitar $f$ com probabilidade pelo menos $1 - 2\lambda\varepsilon$ e daí sabemos que $f$ está $\lambda\varepsilon$-próxima de uma função ditador.Suponha que $i \not\in S$. Por $\chi_{i}(\mathbbm{1}_{S})$ ser 1 e $f$ estar $\lambda\varepsilon$-próxima de $\chi_{i}$:

\begin{equation*}
	\Pr[\text{o método de correção local retorna } 1] \geq 1 - \lambda\varepsilon.
\end{equation*}

Por outro lado, o segunto teste também aceita com probabilidade pelo menos $1 - 2\lambda\varepsilon$ e portanto:

\begin{equation*}
	\Pr[\text{o método de correção local retorna } 1] \leq 2\lambda\varepsilon.
\end{equation*}

Então temos que $2\lambda\varepsilon \geq 1 - \lambda\varepsilon$.

\end{proof}