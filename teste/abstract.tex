\begin{abstract}

\setlength{\parskip}{2pt}
\setlength{\parindent}{1.0cm}

Podemos dizer que complexidade computacional busca descobrir o quão dificil é resolver problemas computacionais. Por exemplo, uma forma de descrever o problema em aberto mais importante da teoria da computação, $\P \stackrel{?}{=} \NP$, é perguntar se o problema da sastifazibilidade booleana necessita de tempo ``mais do que polinomial'' para ser decidido no caso geral. No entanto, até agora não se obteve muito sucesso em provar limites inferiores para complexidade de problemas.

Classificar problemas pela sua complexidade de circuitos é uma duas principais frentes de pesquisa para provar limites inferior de problemas computacionais e por muitos anos pesquisadores acreditaram que complexidade de circuitos era a chave para provar problemas como $\P$ $\stackrel{?}{=}$ $\NP$, onde a complexidade de circuito de um problema é basicamente o número mínimo de portas lógicas necessárias para implementar um circuito que decida este problema.

Nós veremos que as técnicas conhecidas até recentemente para provar limites inferiores são limitadas e não são suficientes para provar que $\P$ $\neq$ $\NP$. Entretanto, resultados recentes conseguiram se esquivar destas limitações e abriram caminho para novos tópicos de pesquisa.

A princípio, o objetivo do trabalho é realizar um estudo sobre os resultados mais recentes em complexidade de circuitos.
\\[.5cm]
\textbf{Palavras chaves:} complexidade computacional, complexidade de circuitos.

\end{abstract}
